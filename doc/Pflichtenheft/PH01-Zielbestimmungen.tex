\section{Zielbestimmungen}
\label{sec:zielbestimmungen}

\textit{Vibora.de - die App} ist ein RSS-Feed Reader f�r aktuelle Android-Ger�te,
der die RSS-Feeds von der Webseite Vibora.de darstellt. Sobald ein neuer
Feed ver�ffentlicht wird, wird dies ein Nachrichten-Dienst von \textit{Vibora.de - die App}
erkennen und eine Nachricht dem Benutzer gezeigt.

\subsection{Musskriterien}

\begin{itemize}
  \item Feed-Liste
    \begin{itemize}
      \item Neue Feeds werden oben in der Liste gezeigt.
      \item Ein Feed zeigt Titel, Datum, Teaser-Text und Link an.
      \item Die Feeds werden als Liste dargestellt.
      \item Der Benutzer kann Feeds aus der Liste l�schen.
    \end{itemize}
  \item Nachricht
    \begin{itemize}
      \item Der Nachrichten-Dienst erzeugt eine Nachricht.
      \item Die Nachricht besteht aus: App-Icon, Titel des Feeds und dem
      Teaser-Text.
      \item Ein Klick auf die Nachricht l�scht diese und �ffnet die App.
    \end{itemize}
  \item Optionen
    \begin{itemize}
      \item Der Benutzer kann angeben, in welchem Intervall der 
      Nachrichten-Dienst nach neuen Feeds schaut.
      \item Der Nachrichten-Dienst ist ein- und ausschaltbar.
      \item Eine Benachrichtigungs-Farbe kann eingestellt werden.
    \end{itemize}
  \item Sonstiges
    \begin{itemize}
      \item In der ActionBar muss das Launcher-Icon und ein
      Icon f�r die \textit{Optionen} sein.
      \item Zu den wichtigsten Methoden sollen Unit-Tests erstellt sein.
      \item Ein StrictMode-Test muss fehlerfrei laufen.
      \item SharedPreferences und PreferenceFragment muss genutzt werden.
      \item Ein Dialog fragt den Benutzer, falls die App
      mit der \textit{Zur�ck-Taste} beendet wird.
      \item Die App ist kostenlos, OpenSource mit BSD-2 Lizenz.
      \item Der Code muss mit JavaDoc dokumentiert sein.
      \item Zu Testzwecken kann in den Einstellungen auch eine andere
      RSS-Feed URL angegeben werden.
    \end{itemize}
\end{itemize}

\subsection{Wunschkriterien}

\begin{itemize}
  \item Beim Ersten Start der App wird der Benutzer gefragt, ob beim
  Starten des Handys ein Nachrichten-Dienst gestartet werden soll.
  \item Das Starten des Nachrichten-Dienstes beim Starten des Handys 
  ist ein- und ausschaltbar.
  \item Der Benutzer kann die App umschalten auf Deutsch, Englisch.
  \item Der Benutzer kann einzelne Feeds als Favoriten abspeichern.
  \item Der Benutzer hat die Wahl zwischen einem hellen und einem 
  dunklen Layout.
  \item Die App erscheint im F-Droid Store.
\end{itemize}

\subsection{Abgrenzungskriterien}

\begin{itemize}
  \item Ein Login oder �hnliches um z.B. auf den RSS-Feed zu greifen 
  zu k�nnen, ist nicht n�tig.
  \item Andere XML oder RSS-Feed Formate, als der von Vibora.de 
  (RSS 2.0) m�ssen nicht unterst�tzt werden.
  \item Die Darstellung der Feeds in der App sind nur textual.
  \item Die App wird nicht im Amazone oder Google-Play Store erscheinen
  Store ver�ffentlicht.
\end{itemize}
