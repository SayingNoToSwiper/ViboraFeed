\section{Produktfunktionen}

\subsection{Benutzerfunktionen}

\subsubsection{Feed-Liste}

Beim Starten der App sieht der Benutzer zuerst die Feed-Liste.

\begin{description}
  \item[/F0010/] \textit{Liste:} Alle aktuellen Feeds werden in einer Liste dargestellt.
  \item[/F0011/] Die Liste ist scrollbar.
  \item[/F0012/] Links in den Feeds werden in einer Web-Browser App ge�ffnet.
  \item[/F0013/] \textit{ActionBar:} Icon, Anwendungsname, Options-Icon
\end{description}

\subsubsection{Optionen}

Beim Klicken in der Feed-Liste auf ein Options-Icon in der Aktion-Bar,
werden die \textit{Optionen} ge�ffnet.

\begin{description}
  \item[/F0110/] \textit{Dienst}: Der Nachrichten-Dienst, der den Benutzer
  auf neuen Feeds hinweist, kann hier ein- und ausgeschaltet werden.
  \item[/F0111/] \textit{Aktualisierungsintervall:} Der 
  Aktualisierungsintervall des Dienst kann hier angegeben
  werden: 30 Sec, 5 Min, 30Min, 3h
  \item[/F0112/] \textit{Testzwecke:} Die URL des RSS-Feed kann hier eingegeben werden.
  \item[/F0113/] Der Zur�ck-Button f�hrt zur Feed-Liste.
  \item[/F0114/] Eine Benachrichtigungs-Farbe kann eingestellt werden.
\end{description}

\subsubsection{Nachricht}

Der Nachrichten-Dienst erzeugt auf dem Handy eine Nachricht, sobald ein
neuer Feed gefunden wird.

\begin{description}
  \item[/F0210/] Ein Klicken auf die Nachricht �ffnet die App.
\end{description}

\subsubsection{Beenden Dialog}

\begin{description}
  \item[/F0310/] \textit{Benutzer klickt auf ZUR�CK in der Feed-Liste:} Der
  Benutzer wird gefragt, ob er die Anwendung Beenden will.
  \item[/F0311/] W�hlt der Benutzer JA wird, die Anwendung beendet.
  \item[/F0312/] W�hlt der Benutzer NEIN wird, wird zur Anwendung zur�ckgekehrt.
  \item[/F0313/] \textit{Nachricht:} Egal, wof�r sich der Anwender entscheidet,
  der Nachrichten Dienst beleibt von der Wahl unber�hrt.
\end{description}

